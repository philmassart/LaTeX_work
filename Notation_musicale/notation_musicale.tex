% XeLaTeX can use any Mac OS X font. See the setromanfont command below.
% Input to XeLaTeX is full Unicode, so Unicode characters can be typed directly into the source.

% The next lines tell TeXShop to typeset with xelatex, and to open and save the source with Unicode encoding.

%!TEX TS-program = XeLaTeX
%!TEX encoding = UTF-8 Unicode

\documentclass[11pt, twocolumn]{scrreprt}

\usepackage{flupstyleutfxelatex}

%\usepackage{geometry}                % See geometry.pdf to learn the layout options. There are lots.
\geometry{a4paper}                   % ... or a4paper or a5paper or ... 
%\geometry{landscape}                % Activate for for rotated page geometry
%\usepackage[parfill]{parskip}    % Activate to begin paragraphs with an empty line rather than an indent
\usepackage{graphicx}
\usepackage{amssymb}

% Will Robertson's fontspec.sty can be used to simplify font choices.
% To experiment, open /Applications/Font Book to examine the fonts provided on Mac OS X,
% and change "Hoefler Text" to any of these choices.

\usepackage[frenchb]{babel}

\usepackage{fontspec,xltxtra,xunicode}
%\usepackage{fontspec,xunicode,luatextra}

%\defaultfontfeatures{Mapping=tex-text}
\defaultfontfeatures{Ligatures=TeX}

%\setromanfont[Mapping=tex-text]{Garaline Regular}
%\setromanfont{Linux Libertine O}
%\setsansfont[Scale=MatchLowercase,Mapping=tex-text]{Gill Sans}
%\setmonofont[Scale=MatchLowercase]{Andale Mono}

%\defaultfontfeatures{Mapping=tex-text, Fractions=On} 
%\defaultfontfeatures{Mapping=tex-text} 
%\setromanfont [Ligatures={Common}, Numbers={OldStyle}, Variant=02]{Linux Libertine O} 
%\setromanfont [Ligatures={Common}, Numbers={OldStyle}, Variant=01]{Linux Libertine O}
\setromanfont{Linux Libertine O}
\setsansfont{Linux Biolinum O}

\usepackage{polyglossia}
\setdefaultlanguage{french}

\usepackage{tikz}
%\usepackage{fetatolatex}
\usepackage{fancyhdr}
\usepackage{lilyglyphs}
\newcommand{\coda}{\lilyGlyph[scale=1.4, raise=0.5]{scripts.coda}}
\newcommand{\segno}{\lilyGlyph[scale=1.4, raise=0.5]{scripts.segno}}

\usepackage{bchart}

\usepackage{microtype}
\usepackage{multirow}
%\usepackage[french=guillemets*]{csquotes}
%\MakeOuterQuote{"}
\frenchspacing
\usepackage{hyperref}
%\newcommand{\ieme}{\textsuperscript{ieme}}
%\newcommand{\iere}{\textsuperscript{iere}}
\setlength{\parindent}{0pt} %Supprime l'indentation en début de paragraphe
\usepackage{makeidx}
\makeindex
\FrenchFootnotes
%\usepackage{libertineotf}

% taking care of orphans/widows 
\widowpenalty=10000 
\clubpenalty=10000 
\raggedbottom
\sloppy

\title{Brief Article}
\author{The Author}
%\date{}                                           % Activate to display a given date or no date

\usepackage{soul}
\sethlcolor{black}
\usepackage{color}
\usepackage{titlesec}


\titleformat{\section}
    {\titlefont\color{white}}
    {\hl{\thesection{\kern.66em}}}
    {0pt}
{\hl}

\usepackage{criticalreport}

\begin{document}


%\thispagestyle{empty}
%\pagenumbering{roman}
%\begin{center}
%\begin{LARGE}
%Académie de Woluwe-Saint-Pierre
%\end{LARGE}
%\end{center}
%
%\begin{center}
%\begin{LARGE}
%\textbf{Cours de formation musicale}
%\end{LARGE}
%\end{center}
%
%\begin{center}
%\vspace{5cm}
%\noindent\rule{\textwidth}{0.5mm}
%\begin{huge}
%\textbf{Aide-mémoire de notation musicale \\
%%\vspace{2 cm}
% %Qualification [Adultes] 2
% }
%\end{huge}
%\noindent\rule{\textwidth}{0.5mm}
%\end{center}

%------------------------------------INTRODUCTION--------------------
%------------------------------------------------------------------------------
%\clearpage
%\thispagestyle{empty}
%\pagenumbering{roman}
%\addcontentsline{toc}{chapter}{\numberline{}Remerciements}
%\chapter*{Remerciements}
%\begin{vcenterpage}


%\paragraph{À l'attention des parents}
%\paragraph{}
%Ce petit aide-mémoire reprend les différents notions de théorie correspondantes à chaque degré de formation musicale à l'académie. Chaque professeur expliquera ces notions à sa manière devant les élèves, mais les pages qui suivent permettent de retrouver rapidement les différents chapitres vus pendant l'année, de façon résumée mais néanmoins précise.


%\end{vcenterpage}
%--------------------------TDM ETC------------ ------------------------
%------------------------------------------------------------------------------
\newpage
\pagenumbering{roman}
%\selectlanguage{french}
\lhead{Aide-mémoire de notation musicale}
%\rhead{Qualification Adultes 2}
\fancyfoot[CO]{\thepage}%ajouter CE si a marche pas
\tableofcontents
%\sommaire
%\listoffigures
\cleardoublepage
%--------------------------CHAPITRE 1------------ ----------------
%------------------------------------------------------------------------------
%\chapter{Clés, intervalles}
\clearpage
\pagenumbering{arabic}
\lhead{Aide-mémoire de notation musicale}
%\rhead{Qualification Adultes 2}
\fancyfoot[CO]{\thepage}%ajouter CE si ça marche pas

%\chapter{Clés, intervalles}
%\section{Les clés}
%Il existe 3 types de clés différents: 
%\begin{description}
%\item [clé de sol:] (2\ieme{} ligne)
%\item [clés de fa:] (4\ieme{} ou 3\ieme{} ligne)
%\item  [clés d'ut:] (1\iere, 2\ieme, 3\ieme{} ou 4\ieme{} ligne)
%\end{description}
%\begin{center}
%\includegraphics{./exemples/ex_cles.pdf}
%\end{center}

\section*{A cappella}
Passage de musique vocale qu'il faut exécuter \textbf{sans accompagnement} (a cappella). Une partie d'accompagnement est parfois prévue pour les répétitions; cette partie sera alors écrite en plus petites notes.

\input{annotate.annotations.inp}

\section*{A due (a2)} Voir \emph{Partage de portée}

\section*{A tempo}
Indication en haut de la portée pour indiquer un retour au tempo précédent; souvent utilisé après des changements tels que \emph{ritardando}, \emph{accelerando}, \emph{più lento} etc. {Voir Indications de tempo}

\section*{Accents} Voir articulations

\section*{Altérations}
Altérations temporaires de la hauteur d'une note

\subsection*{Les cinq symboles principaux}
S'ils apparaissent dans une partition -- à l'exception de l'armure --, ces symboles sont appelés \emph{altérations}.
\begin{enumerate}
\item[\sharp] Dièse: monte la note d'un demi-ton
\item[\doublesharp] Double dièse: monte la note de deux demi-tons
\item[\flat] Bémol: abaisse la note d'un demi-ton
\item[\flatflat] Double bémol: abaisse la note de deux demi-tons
\item[\natural] Bécarre: annule une des altérations ci-dessus, qu'elle soit écrite à l'armure ou en cours de partition
\end{enumerate}


%\addcontentsline{toc}{chapter}{\numberline{}Index}
%\printindex
\end{document}



\end{document}  