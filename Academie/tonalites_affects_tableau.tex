\renewcommand*{\arraystretch}{1.5}
\begin{table}
	\begin{center}
		\begin{tabularx}{15cm}{| l | X | X |}
		\hline
			\centering{\textbf{TONALITÉ}} & \centering{\textbf{Majeur}}                                                                    & \centering\arraybackslash{\textbf{Mineur}}                                                                                           \\ 
		\hline
			DO       & Insolence, effronterie, allégresse, joie                                  & Amabilité, tristesse, douceur                                                                     \\ 
		\hline
			RÉ       & Vivacité, gaieté, caprice, délicatesse, entêté, bruyant, réconfortant     & Calme, moelleux, noblesse, pieux                                                                  \\ 
		\hline
			MI \flat & Pathétique, sérieux, divin (Trinité)                                      &                                                                                                   \\ 
		\hline
			MI       & Désespoir, tranchant et pénétrant, en souffrant, avec affliction mortelle & Pensée profonde, affection, tristesse mais avec consolation, légèreté sans joie                   \\ 
		\hline
			FA       & Générosité, constance, amour, naturel (Pastorale), aisé, beau             & Mélancolie noire sans secours, avec angoisse de mort, résigné                                     \\ 
		\hline
			FA \sharp     &                                                                           & Grande affliction, misanthropie                                                                   \\ 
		\hline
			SOL      & Sérieux, gaieté, insinuant, racontant                                     & Sérieux, grâce, beauté extraordinaire, gaiement et doux, gracieux et tendre, avec plainte adoucie \\ 
		\hline
			LA       & Tristesse, plainte, aussi brillant                                        & Plainte, calme, invitant au sommeil                                                               \\ 
		\hline
			SI \flat & Divertissement, magnificence, très agréable                               & Sensibilité et tristesse                                                                          \\ 
		\hline
			SI       & Désespoir, dureté                                                         & Mélancolie, mauvaise humeur, sans plaisir            \\                                            
		\hline
		\end{tabularx}
		\caption{Les tonalités et leurs affects, d'après \textsc{J. Mattheson}, \emph{Das Neu-Eröffnete Orchester}, 1713}
		\label{}
	\end{center}
\end{table}