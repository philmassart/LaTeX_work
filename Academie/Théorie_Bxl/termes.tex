%!TEX root = theorie_solfege_Bxl.tex
\chapter{Termes musicaux}

Voici quelques termes musicaux ainsi que leur signification. Ces termes sont souvent en italien.

\section{Nuances}\index{nuances}
\begin{description}
\item \lilyDynamics{p} : doux, doucement (\emph{piano} en italien)
\item \lilyDynamics{f} : fort (\emph{forte} en italien)
\item \lilyDynamics{mf} : modérément fort (\emph{mezzo forte} en italien)
\item \lilyDynamics{mp}: modérément doux (\emph{mezzo piano} en italien)
\item Crescendo\index{crescendo} \crescHairpin{}: en augmentant l'intensité du son
\item Decrescendo\index{decrescendo} \decrescHairpin{}: en diminuant l'intensité du son
\end{description}

\section{Tempo et mesure}
\begin{description}
\item Lento: lent
\item Adagio: à l'aise
\item Andante: allant
\item Moderato: modéré
\item Allegro: rapide, joyeux
\item \lilyTimeC{}: mesure \lilyTimeSignature{4}{4}\index{\lilyTimeC{}}
\end{description}

\section{Divers}
\begin{description}
\item \fermata\index{point d'orgue} : point d'orgue. Prolonge une note ou un silence (durée au choix du musicien). Sur un silence, on parle de point d'arrêt. On peut aussi l'indiquer \fermatalong
\item Legato\index{legato} (lié): les notes sont jouées sans interruption du son entre elles.
\item \includegraphics{exemples/legato.pdf}
\item Staccato\index{staccato} (piqué): les notes sont jouées de façon courte. Le point est placé sous ou au-dessus de la note (à l'opposé de la hampe)
\item \includegraphics{exemples/staccato.pdf}
\item Anacrouse\index{anacrouse}: mesure incomplète au début d'un morceau de musique
\item Da capo (D. C.): reprendre au début 
\item Équisonance\index{equisonance@équisonance} ou enharmonie\index{enharmonie}: note de même son, mais de nom différent. Exemples: si~--~do~\flat{} ou fa~\sharp{} -- sol \flat.
\item Poco a poco: peu à peu, petit à petit
\item Tessiture\index{tessiture}: c'est l'ensemble des notes qu'un instrument peut jouer, du grave jusqu'à l'aigu. Elle permet de dire si un instrument est grave ou aigu.

\end{description}