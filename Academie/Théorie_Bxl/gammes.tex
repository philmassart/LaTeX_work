%!TEX root = theorie_solfege_Bxl.tex
\chapter{Gammes, tonalités}
\section{Gammes}\index{gamme}
Une gamme est une suite d'intervalle. Il existe de nombreuses gammes et le nombre de notes qu'elles contiennent n'est pas toujours le même. Cependant, il existe 2 gammes très connues (gamme majeure et gamme mineure) et elles comptent chacune 8 notes, donc 7 différentes (la 8\ieme{} porte le même nom que la première).
\subsection{La gamme majeure}\index{gamme majeure}
C'est la gamme la plus connue. Elle est fabriqué à partir d'une suite de tons et de demi-tons dans cet ordre:
\begin{center}
\fbox{1t - 1t - $\frac 1 2$ t - 1t - 1t - 1t - $\frac 1 2$ t}
\end{center}
\subsubsection{Do Majeur}

Voici l'exemple de Do Majeur.\index{do majeur}
\begin{center}
\includegraphics{exemples/gamme_maj.pdf}
\end{center}

Il existe beaucoup d'autres gammes majeures. Pour respecter l'ordre des tons et demi-tons dans la gamme, il faut monter ou abaisser certaines notes, en utilisant des \sharp{} ou \flat. Voici 2 exemples:
\subsubsection{Sol Majeur}\index{sol majeur}
\begin{center}
\includegraphics{exemples/sol_maj.pdf}
\end{center}

\subsubsection{Fa Majeur}\index{fa majeur}
\begin{center}
\includegraphics{exemples/fa_maj.pdf}
\end{center}

\subsection{La gamme mineure}
La gamme mineure possède plusieurs formes couramment utilisées. Toutes sont malgré tout basées sur la gamme mineure \emph{antique}, qui est la seule à correspondre à son armure.

\subsubsection{Gamme mineure antique\label{min_ant}}\index{gamme mineure antique}
\begin{center}
\includegraphics{exemples/gamme_min_ant.pdf}
\end{center}

Elle est construite sur le schéma suivant: 
\begin{center}
\fbox{1t - $\frac 1 2$ t - 1t - 1t - $\frac 1 2$ t - 1t - 1t}
\end{center}
Le 7\ieme{} degré est ici appelé \emph{sous-tonique}\index{sous-tonique} et non \emph{sensible}, vu qu'il y a 1 ton entre cette note et la tonique.

\subsubsection{Gamme mineure harmonique}\index{gamme mineure harmonique}
\begin{center}
\includegraphics{exemples/gamme_min_har.pdf}
\end{center}

Pour palier à l'absence de sensible, cette gamme présente un 7\ieme{} degré haussé, devenant ainsi une sensible. Par contre, l'intervalle VI - VII d'1 ton $\frac1 2$ (seconde augmenté) rend cette gamme mélodique. Son nom  \og harmonique \fg{} indique d'ailleurs qu'elle est utilisé pour constituer l'harmonie et les accords d'une musique.

Elle est construite sur le schéma suivant: 
\begin{center}
\fbox{1t - $\frac 1 2$ t - 1t - 1t - $\frac 1 2$ t - 1t $\frac 1 2$ - $\frac 1 2$ t}
\end{center}

\subsubsection{Gamme mineure mélodique}\index{gamme mineure mélodique}
Tout en conservant la sensible introduite dans la gamme mineure harmonique, elle rétablit l'intervalle VI - VII d'1 ton, rendant la gamme beaucoup plus adaptée à l'écriture mélodique. Cette gamme, contrairement aux autres possède une forme descendante différente de la forme ascendante (montante), et qui est similaire à la gamme antique.

Elle est construite sur le schéma suivant: 
\begin{center}
\fbox{1t - $\frac 1 2$ t - 1t - 1t - 1t - 1t  - $\frac 1 2$ t}
\end{center}
\begin{center}
\includegraphics{exemples/gamme_min_mel.pdf}
\end{center}
%D'autres gammes sont souvent associées aux gammes mineures (gamme mixte, gamme bohémienne, gamme  orientale) mais il est plus logique de les considérer comme des modes à part entière\footnote{L'aberration consistant à considérer la gamme mixte comme gamme mineure -- alors que son accord parfait est majeur -- est un exemple parmi d'autres.}.

%Les autres modes ont malgré tout gardé une certaine importance, variable selon les époques et les lieux (musique populaire, jazz, musiques du monde etc.).

\subsection{L'accord parfait}\index{accord parfait}
L'accord parfait d'une gamme est composé de la 1\iere{} (tonique), la 3\ieme{} (médiante) et la 5\ieme{} (dominante) note d'une gamme. Jouées l'une après l'autre, ces notes forment un arpège\index{arpège}.


\section{Degrés}\index{degrés}
Chaque note dans la gamme porte un nom, en fonction de sa place. On numérote les notes de la gamme en utilisant les chiffres romains.
\begin{itemize}
\item la première (I) s'appelle \textbf{tonique}\index{tonique}
\item la deuxième (II) s'appelle \textbf{sus-tonique}\index{sus-tonique}
\item la troisième (III) s'appelle \textbf{médiante}\index{médiante}
\item la quatrième (IV) s'appelle \textbf{sous-dominante}\index{sous-dominante}
\item la cinquième (V) s'appelle \textbf{dominante}\index{dominante}
\item la sixième(VI) s'appelle \textbf{sus-dominante}\index{sus-dominante}
\item la septième (VII) s'appelle \textbf{sensible}\index{sensible}
\end{itemize}
La huitième note (VIII) est la même que la première.

Les plus importants à retenir sont la tonique, la dominante et la sensible.

En mineur antique (voir \ref{min_ant} page \pageref{min_ant}), la septième note de la gamme porte le nom de sous-tonique, étant donné qu'il y a un ton au lieu d'$\frac1 2$ entre cette dernière et la tonique. Les degrés de la gamme se notent de chiffres romains.

\section{Armures}
Les altérations présentes à l'armure se présentent toujours dans un ordre précis.
\subsubsection{Ordre des dièses}\index{ordre des dièses}
FA DO SOL RÉ LA MI SI
\subsubsection{Ordre des bémols}\index{ordre des bémols}
SI MI LA RÉ SOL DO FA
\\
\\
Cette armure permet de déterminer la tonalité majeure ou mineure présente dans une partition. À chaque armure correspondent 2 tonalités: une majeure, l'autre mineure. Par facilité, on calcule d'abord la possibilité majeure, avant de calculer la relative mineure.

\subsubsection{Tonalités en dièse}
Le dernier dièse présent à l'armure correspond à la sensible de la tonalité majeure. Il suffit donc de monter d'$\frac1 2$ ton diatonique à partir de cette note (qui est toujours un \sharp) pour obtenir le nom de la tonalité majeure.
\subsubsection{Tonalités en bémols}
La tonique correspond à l'avant-dernier \flat. Exception: Fa Majeur, dont l'armure est si \flat{} (puisqu'il n'y a pas dans ce cas d'avant-dernier \flat).

On peut imaginer les différentes tonalités formant une horloge: dans le sens des aiguilles d'une montre, on ajoute un \sharp; dans le sens contraire, on ajoute un \flat.
 %%%%%%%%%%CYCLE DES QUINTES%%%%%%%%%%%
\begin{center}
\begin{tikzpicture} [line cap=round,line width=1pt] 
\draw (0,0) circle (4.2cm);

\foreach \angle / \label in 
{0/{LA}, 30/{RÉ}, 60/{SOL}, 90/{DO}, 120/{FA}, 150/{SI \flat}, 180/{MI \flat},
210/{LA \flat}, 240/{DO \sharp{} - RÉ \flat}, 270/{FA \sharp{} - SOL \flat}, 300/{SI}, 330/{MI}}
{
\draw[line width=1pt] (\angle:4.0cm) -- (\angle:4.4cm); 
\draw (\angle:4.8cm) node{\label};
}
\foreach \angle / \label in 
{0/{3 \sharp}, 30/{2 \sharp}, 60/{1 \sharp}, 90/{$-$}, 120/{1 \flat}, 150/{2 \flat}, 180/{3 \flat},
210/{4 \flat}, 240/{7 \sharp{} - 5 \flat}, 270/{6 \sharp{} - 6 \flat}, 300/{5 \sharp{} - 7 \flat}, 330/{4 \sharp}}
{
\draw (\angle:3.6cm) node{\textsf{\label}};
}
\node (0,0){Gammes majeures};
\end{tikzpicture}
\end{center}

\begin{center}
\begin{tikzpicture} [line cap=round,line width=1pt] 
\draw (0,0) circle (4.2cm);

\foreach \angle / \label in 
{0/{fa \sharp}, 30/{si}, 60/{mi}, 90/{la}, 120/{ré}, 150/{sol}, 180/{do},
210/{fa}, 240/{la \sharp{} - si \flat}, 270/{ré \sharp{} - mi \flat}, 300/{sol \sharp}, 330/{do \sharp}}
{
\draw[line width=1pt] (\angle:4.0cm) -- (\angle:4.4cm); 
\draw (\angle:4.8cm) node{\label};
}
\foreach \angle / \label in 
{0/{3 \sharp}, 30/{2 \sharp}, 60/{1 \sharp}, 90/{$-$}, 120/{1 \flat}, 150/{2 \flat}, 180/{3 \flat},
210/{4 \flat}, 240/{7 \sharp{} - 5 \flat}, 270/{6 \sharp{} - 6 \flat}, 300/{5 \sharp{} - 7 \flat}, 330/{4 \sharp}}
{
\draw (\angle:3.6cm) node{\textsf{\label}};
}
\node (0,0){Gammes mineures};
\end{tikzpicture}
\end{center}
